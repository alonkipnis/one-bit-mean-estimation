\section{Proof of Theorem~\ref{thm:sgd}}
\label{proof:sgd}

The estimation algorithm~\eqref{eq:sgd_alg} is a special
case of the stochastic gradient procedures in the papers
\cite{PolyakJu92, polyak1990new}.
We rely on several of their results. Throughout this proof,
we assume without loss of generality that the median
$\theta = \mbox{med}(P) = 0$.

\subsection{Proof of Theorem~\ref{thm:sgd}\eqref{item:normal-sgd}}
\label{sec:proof-normal-sgd}

Consider the following simplified version of
\cite[Thm. 4]{polyak1992acceleration}:
\begin{corollary}{\cite[Thms. 3 \& 4]{PolyakJu92}}
  \label{corollary:polyak-juditsky}
  Let $\varphi : \R \to \R$ and $Z_i$ be i.i.d.\ random
  variables, and
  \begin{equation*}
  X_i = \theta + Z_i.
  \end{equation*}
  Define
  \begin{align}
    \begin{split}
      \theta_i & = \theta_{i-1} + \gamma_i \varphi(X_i - \theta_{i-1}), \\
      \bar{\theta}_n & = \frac{1}{n} \sum_{i=0}^{n-1} \theta_i, 
    \end{split}
    \label{eq:Polyak_Juditsky_alg}
  \end{align}
  where in addition,
  \begin{enumerate}[(i)]
  \item There exists $K_1$ such that $\left| \varphi(x) \right| \leq
    K_1(1+|x|)$ for all $x\in \R$.
  \item The sequence $\left\{ \gamma_i \right\}_{i=1}^\infty$ satisfies
    condition~\eqref{eqn:lazy-gamma}.
  \item \label{item:zero-gradient}
    The function $\psi(x) \triangleq \ex{ \varphi(x+Z_1)}$
    satisfies $\psi(0) = 0$ and
    $x\psi(x) > 0$ for $x\neq 0$.  Moreover, $\psi$ is differentiable
    at 0 with $\psi'(0) > 0$ and there exists
    $K_2$ and $0 < \lambda \leq 1$ such that
    \begin{equation}
      \label{eqn:local-hessian-psi}
        \left| \psi(x) - \psi'(0)x \right|\leq K_2 |x|^{1+\lambda}.
    \end{equation}
  \item The function 
    $\chi(x) \triangleq \ex{\varphi^2(x+Z_1)}$ is continuous at zero. 
  \end{enumerate}
  Then $\bar{\theta}_n \cas \theta$ and $ \sqrt{n}({\theta}_n - \theta)
  \cd \normal(0,V)$ for
  $V = \frac{ \chi(0)} {\psi'(0)^2}$.
\end{corollary}

Using the notation in Corollary~\ref{corollary:polyak-juditsky}, we set
$\varphi(x) = \sgn(x)$ and $Z_i = X_i - \theta$, where $\theta =
\mbox{med}(P)$. Without loss of generality and for notational convenience,
we assume for the remainder of this derivation that $\theta = 0$.
As a consequence, we have $\mbox{med}(Z) = 0$,
and $\chi(x) = \ex{ \sgn^2(x+Z_1) }= 1$, so
$\chi(0) = 1$. In addition,
\begin{align*}
  \psi(x) & = \ex{ \sgn(x+ Z_1) } =
  P(Z \ge -x) - P(Z < -x) = 1 - 2 P(Z \le -x).
  %%   \int_{-\infty}^\infty \sgn(x+z) f(z) dz \\
  %% & = \int_{-x}^\infty f(z) dz -\int_{-\infty}^{-x} f(z) dz. 
\end{align*}
Using that $P$ has a density $f$ near its median, it follows that $\psi'(x)
= 2f(-x)$ and thus $\psi'(0) = 2f(0) > 0$.  We may now verify that the
conditions in Corollary~\ref{corollary:polyak-juditsky} hold for $\lambda =
1$. Condition~(i) is obvious, and the convexity of $|\cdot|$ gives most of
condition~(iii) excepting inequality~\eqref{eqn:local-hessian-psi}. For
that, note that as $f$ is Lipschitz near 0 with constant $\lip_0(f)$, we
have for small $x$ that
\begin{equation*}
  \psi(x) = 2 \int_0^x f(-t) dt
  = 2 \int_0^x \left[f(0) \pm \lip_0(f) t \right] dt
  = 2 f(0) x \pm \frac{\lip_0(f)}{2} x^2
  = \psi'(0) x \pm \frac{\lip_0(f)}{2} x^2,
\end{equation*}
so that condition~(iii) holds.
As evidently $\chi(0) / \psi'(0)^2 = \frac{1}{4 f(0)^2}$,
Corollary~\ref{corollary:polyak-juditsky} gives
Theorem~\ref{thm:sgd}\eqref{item:normal-sgd}.

\subsection{Proof of Theorem~\ref{thm:sgd}\eqref{item:sgd-regular}}
\label{sec:proof-sgd-regular}

This proof requires somewhat more technicality than the first part of the
theorem, including a brief detour into local asymptotic normality theory,
regular estimators, and quadratic-mean
differentiability~\cite[cf.]{VanDerVaart98}.
We assume without loss of generality that the median of the density
$f$ is 0, so that if $P_\theta$ has density $f(\cdot - \theta)$, the
median of $P_\theta$ is $\theta$.
We begin by recalling the statistical concepts we require.
\begin{definition}
  \label{definition:regular-estimator}
  A sequence of estimators $T_n$ for a parameter $\theta$ in the parametric
  family $\{P_\theta\}_{\theta \in \Theta}$ is \emph{regular at $\theta$} if
  there exists a distribution $Q$ such that for any bounded sequence
  $h_n$,
  \begin{equation*}
    \sqrt{n}(T_n - (\theta + h_n / \sqrt{n}))
    \mathop{\cd}_{P_{\theta + h_n/\sqrt{n}}}
    Q.
  \end{equation*}
\end{definition}
\begin{definition}
  \label{definition:qmd}
  Let $\{P_\theta\}_{\theta \in \Theta}$ have densities $p_\theta$
  with respect to a base measure $\mu$. The family
  is \emph{quadratic mean differentiable} (QMD) at
  $\theta$ with score $\score_\theta$ if
  \begin{equation}
    \label{eqn:qmd}
    \int \left(\sqrt{p_{\theta + h}} - \sqrt{p_\theta}
    - \half h^\top \score_\theta\sqrt{p_\theta} \right)^2 d\mu
    = o(\norm{h}^2)
  \end{equation}
  as $h \to 0$.
\end{definition}
\begin{definition}
  \label{definition:lan}
  A family of distributions $\{P_\theta\}_{\theta \in \Theta}$ is
  \emph{locally asymptotically normal with information matrix $I_\theta$} (LAN)
  at
  $\theta$ if there exists a sequence of random vectors $Z_n$ such that for
  all $h_n \to h$,
  \begin{equation*}
    \sum_{i = 1}^n \log \frac{dP_{\theta + h_n/\sqrt{n}}}{dP_\theta}
    (X_1, \ldots, X_n)
    = h^\top Z_n - \half h^\top I_\theta h + o_P(1)
  \end{equation*}
  where $Z_n \cd \normal(0, I_\theta)$ under $P_\theta$, where
  $X_i \simiid P_\theta$.
\end{definition}

These three definitions are linked in our case by a few important results.
First~\cite[Theorem 7.2]{VanDerVaart98}, if $\{P_\theta\}$ is QMD
(Def.~\ref{definition:qmd}) at the point $\theta$, then it is locally
asymptotically normal with $Z_n = \frac{1}{\sqrt{n}} \sum_{i = 1}^n
\score_\theta(X_i)$ and information matrix $I_\theta =
\E_\theta[\score_\theta \score_\theta^\top]$. Moreover, in any family
$\{P_\theta\}$ that is LAN (Def.~\ref{definition:lan}) at $\theta$, if $T_n$
is a regular estimator (Def.~\ref{definition:regular-estimator}) at $\theta$
with limiting distribution $Q$, then for any bounded, symmetric,
quasi-convex loss $L$ and $c < \infty$,
\begin{equation}
  \label{eqn:limit-law-regular}
  \limsup_n \sup_{\norm{h} \le c}
  \E_{P_{\theta + h/\sqrt{n}}}
  \left[L(\sqrt{n}(T_n - \theta - h / \sqrt{n}))\right]
  = \E[L(W)] ~~ \mbox{for~}W \sim Q
\end{equation}
(see Beran~\cite{beran1995role}, Eq.~(4.2)).
%
Thus, we show two results: first,
that the family $\{P_\theta\}$ of distributions
defined by the shifted densities $\{f(\cdot - \theta)\}_{\theta \in \R}$
is quadratic-mean-differentiable at any $\theta$,
and second, that $\bar{\theta}_n$ is regular and
asymptotically normal.
The combination evidently gives the theorem.

For quadratic mean differentiability, we have the following lemma, somewhat
more general than we need; we defer proof to Sec.~\ref{sec:proof-qmd}.
\begin{lemma}[Extension of \cite{VanDerVaart98}, Lemma 7.6]
  \label{lemma:qmd}
  Let $p_\theta$ be a density with respect to $\mu$,
  and assume that $\theta \mapsto s_\theta(x) \defeq \sqrt{p_\theta(x)}$
  is absolutely continuous for all $x$. Let
  $\dot{p}_\theta(x) = \nabla_\theta p_\theta(x)$ (when it exists),
  and assume that
  \begin{equation*}
    \mu(\{x : \dot{p}_\theta(x) ~ \mbox{fails~to~exist}\}) = 0.
  \end{equation*}
  Assume that $I_\theta \defeq \E_{P_\theta}[\dot{p}_\theta
    \dot{p}_\theta^\top / p_\theta^2]$ is continuous at $\theta_0$. Then
  $P_\theta$ is QMD (Definition~\ref{definition:qmd}) at $\theta = \theta_0$
  with $\score_\theta = \dot{p}_\theta / p_\theta$.
\end{lemma}

By the assumption in Theorem~\ref{thm:sgd} that the density $f$
is Lipschitz continuous, we see that the location family
$\{P_\theta\}_{\theta \in \R}$ defined by $dP_\theta(x) = f(x - \theta)$
satisfies the conditions of Lemma~\ref{lemma:qmd}.

It remains to show that the average $\bar{\theta}_n$ is regular:
\begin{lemma}
  \label{lemma:sgd-median-regular}
  Let $h_n \to h \in \R$, and define $P_n = P_{\theta + h_n / \sqrt{n}}^n$.
  Then $\bar{\theta}_n$ is regular with
  \begin{align}
    \label{eqn:sgd-median-regular}
    \sqrt{n}\left( \bar{\theta}_n - \theta - h_n / \sqrt{n}\right)
    \mathop{\cd}_{P_n}
    \normal\left( h,\frac{1}{4 f(0)^2}\right).
  \end{align}
\end{lemma}
\begin{proof}
  To show the convergence~\eqref{eqn:sgd-median-regular} we use the
  following refinement of Corollary~\ref{corollary:polyak-juditsky}, which
  provides a generalized convergence result for iteratively defined
  $\theta_n$, and whose
  proof we defer to Section~\ref{proof:normal-expansion}.
  %
  \begin{corollary}
    \label{corollary:normal-expansion}
    Let the conditions of Corollary~\ref{corollary:polyak-juditsky} hold,
    meaning that $\theta_i = \theta_{i-1} + \gamma_i \varphi(X_i -
    \theta_{i-1})$ for $X_i = \theta + Z_i$, where $Z_i$ are i.i.d.\ with
    $\E[\varphi(Z)] = 0$. Additionally assume the local smoothness
    condition that there exist $0 < \lambda \le 1$ and $K < \infty$ such
    that
    \begin{equation}
      \label{eqn:additional-local-smooth}
      \E[|\varphi(x + Z) - \varphi(Z)|^2]
      \le K (|x|^\lambda + x^2).
    \end{equation}
    Set $\Delta_i = \theta_i - \theta$ and $\bar{\Delta}_i = \frac{1}{n}
    \sum_{i=1}^n \Delta_i$. Then
    \begin{enumerate}[(i)]
    \item \label{item:regularity}
      The sequence $\Delta_i$ is regular, that is,
      \begin{equation}
        \sqrt{n} \bar{\Delta}_n
        = -\frac{1}{\sqrt{n}} \frac{1}{\psi'(0)} \sum_{i=1}^{n-1} \varphi(Z_i)
        + o_{P,n}(1).
        \label{eq:normal_expansion_lem}
      \end{equation}
    \item \label{item:apply-le-cam} Let $Z_i$ as in
      Corollary~\ref{corollary:polyak-juditsky} have absolutely continuous
      density $p$ with median $0$, define $\score_h(z) = \frac{p'(z - h)}{p(z
        - h)}$, and assume that $I_h \defeq \E_\theta[\score_h(Z)^2]$ is
      continuous in $h$ near 0.  Then for any converging sequence $h_n \to h$,
      \begin{equation*}
        \sqrt{n} \bar{\Delta}_n
        \mathop{\cd}_{P_{\theta + h_n/\sqrt{n}}^n}
        \normal\left( \frac{-h}{\psi'(0)} \E_p[\varphi(Z) \score_0(Z)],
        \frac{\chi(0)}{\psi'^2(0)} \right).
      \end{equation*}
    \end{enumerate}
  \end{corollary}

  We now verify that the setting of Theorem~\ref{thm:sgd}
  (and Lemma~\ref{lemma:sgd-median-regular}) satisfies the
  conditions of Corollary~\ref{corollary:polyak-juditsky}. First, we have
  the obvious fact that
  \begin{equation*}
    |\sgn(z) - \sgn(x + z)|
    \le 2 \cdot \indic{|x| \ge |z|}.
  \end{equation*}
  Recalling that the density $f$ is Lipschitz with median 0,
  for $\varphi(z) = \sgn(z)$, and $Z = X - \theta$ distributed with
  density $f$, we have
  \begin{align*}
    \ex{ \left| \varphi(Z) - \varphi(x + Z) \right| }
    \leq 2 \Prob\left( |Z_1| \le |x|  \right)
    = 2 \int_{-|x|}^{|x|} f(t) dt
    = 4 f(0)|x| + 2 \int_{-|x|}^{|x|}
    \lip(f) t dt
    = 4 f(0) |x| + 2 \lip(f) x^2
  \end{align*}
  where $\lip(f)$ is the Lipschitz constant of $f$.
  For large $x$ we always have
  $\E[|\varphi(Z) - \varphi(x + Z)|] \le 2$, so that
  condition~\eqref{eqn:additional-local-smooth} holds.
  In addition, we have
  \begin{equation*}
    \int_{\R} \varphi(x) f'(x ) dx
    = \int_{\R} \sgn(x) f'(x ) dx =
    \int_0^\infty f'(x ) dx
    - \int_{-\infty}^0 f'(x ) dx
    = -2 f(0) = -\psi'(0).
  \end{equation*}
  Corollary~\ref{corollary:normal-expansion}
  now implies the convergence \eqref{eqn:sgd-median-regular}.
\end{proof}

Combining Lemmas~\ref{lemma:qmd} and~\ref{lemma:sgd-median-regular}
with the limit~\eqref{eqn:limit-law-regular} gives
Theorem~\ref{thm:sgd}\eqref{item:sgd-regular}.

\subsection{Proof of Theorem~\ref{thm:sgd}\eqref{item:sgd-ms-convergence}}
\label{sec:proof-sgd-ms-convergence}

We begin with the following result from \cite{polyak1990new}:
\begin{corollary}[\cite{polyak1990new}, Theorem 2]
  \label{corollary:polyak-mse}
  Define the iteration
  \begin{align} \label{eq:polyak_new_measurements}
    \begin{cases}
      U_n = U_{n-1} - \gamma_n \varphi(Y_n), & Y_n = g'(U_{n-1})+Z_n \\
      \bar{U}_n= \frac{1}{n} \sum_{i=1}^n U_n, & n=1,2,\ldots.
    \end{cases}
  \end{align}
  Assume that the function $g$ is $\mc{C}^2$, strictly convex, has Lipschitz
  derivative, and is minimized by $x^\star$. Moreover, assume that the
  noises $Z_n$ are i.i.d.\ with density $p$ and
  that the Fisher information $\E[(p'(Z))^2 / p(Z)^2]$ exists and is finite.
  Let $\psi(x)$ and $\chi(x)$ be defined as in
  Corollary~\ref{corollary:polyak-juditsky} and satisfy the conditions in
  the corollary. Assume in addition that $\chi(0)>0$, condition
  \eqref{eqn:local-hessian-psi} with $\lambda = 1$, and there exits $K_3$
  such that
  \begin{equation*}
    \ex{  | \varphi(x+Z_1) |^4 } \leq K_3(1+|x|^4). 
  \end{equation*}
  Finally, assume that the sequence $\{\gamma_n \}$ satisfies conditions
  \eqref{eqn:lazy-gamma} and \eqref{eqn:stringent-gamma}. Then
  \begin{equation*}
    V_n \defeq
    \E\Big[(\bar{U}_n-x^\star )^2\Big] = n^{-1}\frac{\chi(0)} {
      (\psi'(0))^2 (g''(x^\star))^2 } + o(n^{-1}).
  \end{equation*}
\end{corollary}

We apply Corollary~\ref{corollary:polyak-mse} with $g(x) = 0.5(x-\theta)^2$,
$\varphi(x) = \sgn(x)$, $Z_n = \theta-X_n$.
The update~\eqref{eq:polyak_new_measurements} gives
\begin{align*} 
  U_n %% & = U_{n-1} + \gamma_n \sgn(\theta-U_{n-1} - Z_n )  \\
  & = U_{n-1} + \gamma_n \sgn(X_n-U_{n-1} ),
\end{align*}
so the estimator $\bar{U}_n$ is identical to the stochastic gradient
estimator~\eqref{eq:sgd_alg} with $\bar{\theta}_n = \frac{1}{n} \sum_{i =
  1}^n \theta_i$. We have $\E[\varphi(x + Z)^4] = 1$ and by assumption the
Fisher information $\E[(f'(Z))^2 / f(Z)^2]$ exists, and the functions $\psi$
and $\chi$ have the desired conditions of
Corollary~\ref{corollary:polyak-juditsky} (as we verify in
Section~\ref{sec:proof-normal-sgd}). Finally, the function $\theta \mapsto
\E_{P_\theta}[(\bar{\theta}_n - \theta)^2]$ is continuous in $\theta$, so
that for $x^\star = \theta$ and $g'' \equiv 1$, we may apply
Corollary~\ref{corollary:polyak-mse} and a finite approximation to the prior
measure $\pi$ to obtain
Theorem~\ref{thm:sgd}\eqref{item:sgd-ms-convergence}.

%% \begin{equation*}
%% \ex{ | \varphi(x+Z_1) |^4 } = 1 \leq K_3(1+|x|^4)
%% \end{equation*}
%% for any $K_3\geq 1$, the Fisher information of $Z_1$ is $\sigma^2$, $\chi(x) = 1 > 0$, and that 
%% the conditions in Theorem~\ref{thm:polyak_new} on $\psi(x)$ and $\chi(x)$ were verified to hold in the first part of the proof. In particular, $\psi'(0) = (2f(0))^{-2}$. Since $f(x)$ satisfies the conditions above with $x^\star = \theta$ and $g''(x) = 1$. Theorem~\ref{thm:polyak_new} implies that for any $\theta \in \R$, 
%% \begin{equation*}
%% V_n = \ex{ \left({\theta}_n-\theta \right)^2 } = \frac{1}{4n f^2(0)} + o(n^{-1}).
%% \end{equation*}

\subsection{Proof of Corollary~\ref{corollary:normal-expansion}}
\label{proof:normal-expansion}

\subsubsection*{Proof of
  Corollary~\ref{corollary:normal-expansion}\eqref{item:regularity}}

The proof of part~\eqref{item:regularity} requires two additional lemmas of
Polyak and Juditsky~\cite{PolyakJu92}.
\begin{lem}[\cite{PolyakJu92}, Lemma 2]
  \label{lemma:polyak-expansion}
  Define the process $\Delta_i^1 = \Delta_{i-1}^i
  - \gamma_i (A \Delta_{i-1} + \xi_i)$ for $i = 1, 2, \ldots$.
  Assume that $A>0$ and the stepsizes $\gamma_i$ satisfy
  condition~\eqref{eqn:lazy-gamma}. Then
  for $\bar{\Delta}_n^1 = \frac{1}{n} \sum_{i = 1}^n \Delta_i^1$, we have
  \begin{equation}
    \label{eqn:polyak-expansion}
    \sqrt{n} \bar{\Delta}_n^1
    % = \frac{1}{\sqrt{n}}\sum_{i=0}^{n-1} \Delta_i^1
    = \frac{\alpha_n \Delta_0^1}{\sqrt{n} \gamma_0}
    + \frac{1}{\sqrt{n} A} \sum_{i=1}^{n-1} \xi_i
    + \frac{1}{\sqrt{n}}\sum_{i=1}^{n-1} w_i^n \xi_i,
  \end{equation}
  where $\alpha_n$ and $w_i^n$ are real numbers such that $|\alpha_n| \leq
  K$ and $|w_i^n|\leq K$ for some $K< \infty$, and $\lim_{n\to \infty}
  \frac{1}{n} \sum_{i=1}^{n-1} |w_i^n| = 0$.
\end{lem} 

\begin{lem}[\cite{PolyakJu92}]
  \label{lemma:converging-power-sum}
  Under the conditions of Corollary~\ref{corollary:normal-expansion},
  with probability 1,
  \begin{equation*}
  \sum_{i=1}^\infty \frac{|\Delta_{i}|^{1+\lambda}}{\sqrt{i}} < \infty.
  \end{equation*}
\end{lem}
\noindent
Lemma~\ref{lemma:converging-power-sum} follows from the proof of Theorem 2
in \cite[page 851]{polyak1992acceleration}.

We separate the proof of part~\eqref{item:regularity} into two lemmas, which
mirror the proofs of Polyak and Juditsky~\cite{PolyakJu92}; together they
immediately give the result.

\begin{lemma}
  The expansion~\eqref{eq:normal_expansion_lem} holds for the process
  $\bar{\Delta}^1_n$ defined by the iteration
  \begin{align}
    \label{eqn:polyak-expansion_lem1_alg}
    & \Delta_i^1  = \Delta_{i-1}^1 - \gamma_i \psi'(0) \Delta_{i-1}^1 - \gamma_i \varphi(Z_i), \qquad
    \Delta_0^1 = \Delta_0\\
    & \bar{\Delta}^1_n = \frac{1}{n}\sum_{i=0}^{n-1} \Delta^1_i.
    \nonumber
  \end{align}
\end{lemma}
\begin{proof}
  To prove this claim, use Lemma~\ref{lemma:polyak-expansion} with $A =
  \psi'(0)$ and $\xi_i = -\varphi(Z_i)$, which by
  condition~\eqref{item:zero-gradient} in
  Corollary~\ref{corollary:polyak-juditsky} gives that
  $\E[\xi_i] = 0$ and that the $\xi_i$ are independent.
  The first term $\alpha_n \Delta_0^1 / \gamma_0 \sqrt{n} \to 0$
  in Eq.~\eqref{eqn:polyak-expansion}. In addition,
  by independence and that the $\xi_i$ are mean-zero, we have
  \begin{align*}
    \ex{ \left( \frac{1}{\sqrt{n}} \sum_{i=1}^{n-1} w_i^n \xi_i \right)^2 }
    & = \frac{1}{n}  \sum_{i=1}^n (w_i^n)^2 \ex{ \xi_i^2} + \frac{1}{n}  \sum_{i\neq j}^n w_i^n w_j^n \ex{ \xi_i \xi_j} \\
    & = \frac{1}{n}  \sum_{i=1}^n (w_i^n)^2 \ex{ \varphi(Z_i)^2} = \chi(0) \frac{1}{n}  \sum_{i=1}^n (w_i^n)^2 \to 0
  \end{align*}
  by Lemma~\ref{lemma:polyak-expansion}.
  Thus, the expansion~\eqref{eqn:polyak-expansion} in
  Lemma~\ref{lemma:polyak-expansion} gives
  \begin{equation*}
    \sqrt{n} \bar{\Delta}^1_n
    = -\frac{1}{\sqrt{n}} \frac{1}{\psi'(0)}
    \sum_{i=1}^{n-1} \varphi(Z_i)+ o_{P,n}(1)
  \end{equation*}
  as desired.
\end{proof}

We then have the following asymptotic equivalence.
\begin{lemma}
  The sequences $\bar{\Delta}_n$ and $\bar{\Delta}^1_n$ are asymptotically
  equivalent, meaning that
  $\sqrt{n} (\bar{\Delta}_n - \bar{\Delta}_n^1) \cp 0$.
\end{lemma}
\begin{proof}
  From the recursions~\eqref{eq:Polyak_Juditsky_alg} and
  \eqref{eqn:polyak-expansion_lem1_alg}, the difference $\delta_i = \Delta_i
  - \Delta_i^1$ satisfies
  \begin{equation*}
  \delta_i = \delta_{i-1} - \gamma_i \psi'(0) \delta_{i-1} + \gamma_i \left( \psi'(0) \Delta_{i-1}  + \varphi(Z_i) - \varphi(\Delta_{i-1} + Z_i) \right),
  \end{equation*}
  where $\delta_0 = 0$. Applying Lemma~\ref{lemma:polyak-expansion} with the
  choices $\xi_i = \psi'(0) \Delta_{i-1} + \varphi(Z_i) -
  \varphi(\Delta_{i-1} + Z_i)$ yields
  \begin{align}
    \sqrt{n}\bar{\delta}_n
    & = \frac{1}{\sqrt{n}} \sum_{i=1}^{n-1}
    \left( \frac{1}{\psi'(0)} + w_i^n \right)  \xi_i  \nonumber \\
    & = \frac{1}{\sqrt{n}} \sum_{i=1}^{n-1}
    \left( \frac{1}{\psi'(0)} + w_i^n \right)
    \left( \psi'(0) \Delta_{i-1}  - \psi(\Delta_{i-1}) \right)
    \label{eq:PJ_proof1} \\
    & \qquad ~ + 
    \frac{1}{\sqrt{n}} \sum_{i=1}^{n-1} \left( \frac{1}{\psi'(0)}
    + w_i^n \right)
    \left( \psi(\Delta_{i-1})  + \varphi(Z_i) - \varphi(\Delta_{i-1}+Z_i)
    \right) \label{eq:PJ_proof2}
  \end{align}
  For the term \eqref{eq:PJ_proof1},
  the assumption~\eqref{eqn:local-hessian-psi} that
  $|\psi(x) - \psi'(0) x| = O(x^{1 + \lambda})$
  and that $\sup_{i,n} |w_i^n| < \infty$ by Lemma~\ref{lemma:polyak-expansion}
  give that there exists $K < \infty$ such that
  $|\psi'(0)^{-1} + w_i^n| |\psi'(0) \Delta_{i-1} - \psi(\Delta_{i-1})|
  \le K |\Delta_i|^{1 + \lambda}$.
  Lemma~\ref{lemma:converging-power-sum} gives that
  $\sum_{i = 1}^n
  \frac{1}{\sqrt{i}} |\Delta_i|^{1 + \lambda} < \infty$,
  and so the Kronecker lemma gives that
  \begin{equation*}
    \frac{1}{\sqrt{n}} \sum_{i=1}^{n-1} \left( \frac{1}{\psi'(0)}  + w_i^n \right)  \left( \psi'(0) \Delta_{i-1}  - \psi(\Delta_{i-1}) \right) \cas 0.
  \end{equation*}

  The term \eqref{eq:PJ_proof2} is somewhat more challenging to control.
  We define
  \begin{equation*}
    \epsilon_i \triangleq \psi(\Delta_{i-1}) + \varphi(Z_i)
    - \varphi(\Delta_{i-1}+Z_i),
  \end{equation*}
  and let
  $\mc{F}_i = \sigma(Z_1, \ldots, Z_i)$ be the $\sigma$-field of the randomness
  through time $i$. We use a square integrable martingale convergence
  theorem~\cite[Exercise~5.3.35]{Dembo16}. Noting that
  $\Delta_i \in \mc{F}_i$, we have
  \begin{align}
    \E[\epsilon_i^2 \mid \mc{F}_{i-1}]
    & = \E[(\psi(\Delta_{i-1}) + \varphi(Z_i)
      - \varphi(\Delta_{i-1} + Z_i))^2 \mid \mc{F}_{i-1}]
    \nonumber \\
    & \le 2 \psi(\Delta_{i-1})^2
    + 2 \E[(\varphi(\Delta_{i-1} + Z_i) - \varphi(Z_i))^2 \mid \mc{F}_{i-1}]
    \nonumber \\
    & \le
    K \left[|\Delta_{i-1}|^{1 + \lambda}
      + |\Delta_{i-1}|^\lambda
      + \Delta_{i-1}^2 \right],
    \label{eqn:bound-psi-error-expectations}
  \end{align}
  where inequality~\eqref{eqn:bound-psi-error-expectations} follows by the
  conditions~\eqref{eqn:local-hessian-psi}
  and~\eqref{eqn:additional-local-smooth}, and $\E[\varepsilon_i \mid
    \mc{F}_{i-1}] = 0$ for all $i$ by definition of $\psi(x) = \E[\varphi(x
    + Z)]$ and that $\psi(0) = 0$.  We now control the expectations of these
  quantities. For $R < \infty$, define the stopping time $\tau_R \defeq \inf
  \{i : |\Delta_i| > R\}$, which satisfies $\{\tau_R \le i\} \in \mc{F}_i$
  for each $i$.  Then using~\cite[Eq.~(A14)]{PolyakJu92}, we have
  \begin{equation*}
    \E[\Delta_i^2 \indic{\tau_R > i}] \le K \gamma_i,
  \end{equation*}
  and so inequality~\eqref{eqn:bound-psi-error-expectations} gives that
  \begin{equation*}
    \E\bigg[\sum_{i = 1}^\infty \frac{1}{i}
      |\varepsilon_i|^2 \indic{\tau_R > i} \bigg]
    \le K \sum_{i = 1}^\infty \frac{\gamma_i^\lambda}{i}
    < \infty
    ~~ \mbox{so} ~~
    \sum_{i = 1}^\infty \frac{1}{i}
    \varepsilon_i^2 \indic{\tau_R > i} < \infty
    ~ \mbox{a.s.}
  \end{equation*}
  by Condition~\eqref{eqn:lazy-gamma}.
  As with probability 1 there exists some (random) $R < \infty$ such that
  $\tau_R = \infty$, we obtain that 
  \begin{equation*}
    \sum_{i = 1}^\infty \frac{1}{i}
    \varepsilon_i^2  < \infty
    ~ \mbox{a.s.}
  \end{equation*}
  Applying the square integrable martingale convergence
  theorem of \cite[Ex.~5.3.35]{Dembo16}, we have
  \begin{equation*}
    \frac{1}{\sqrt{n}} \sum_{i = 1}^n
    \left(\frac{1}{\psi'(0)} + w_i^n\right) \epsilon_i \cas 0,
  \end{equation*}
  so that both equations~\eqref{eq:PJ_proof1} and~\eqref{eq:PJ_proof2}
  converge almost surely to 0.
\end{proof}

\subsubsection*{Proof of
  Corollary~\ref{corollary:normal-expansion}\eqref{item:apply-le-cam}}

This is essentially an immediate consequence of Le Cam's third
lemma~\cite[Example 6.7]{VanDerVaart98}.
Recall~\cite[Thm.~7.2]{VanDerVaart98} that if
a family $\{P_\theta\}_{\theta \in \Theta}$ is quadratic mean differentiable
at $\theta$ with score $\score_\theta$, then
it is LAN at $\theta$ (Definition~\ref{definition:lan})
with information matrix $I_\theta = \E[\score_\theta \score_\theta^\top]$.

%% then for any sequence
%% $h_n \to h$, if $Z_i \simiid P_\theta$ then
%% \begin{equation}
%%   \label{eqn:lan-from-qmd}
%%   \log \frac{dP_{\theta + h/\sqrt{n}}^n}{dP_\theta} (Z_1, \ldots, Z_n)
%%   = \frac{h}{\sqrt{n}} \sum_{i = 1}^n \score_\theta(Z_i)
%%   - \half h^2 I_\theta
%%   + o_{P_\theta,n}(1)
%% \end{equation}
%% where $I_\theta = \E_{P_\theta}[\score_\theta^2]$ is the Fisher information,
%% which necessarily exists.

%% We thus show that under the conditions we have assumed,
%% we have quadratic mean differentiability.

%% Deferring the proof of Lemma~\ref{lemma:qmd} temporarily,

The regularity result~\eqref{eq:normal_expansion_lem} gives
\begin{equation*}
  \sqrt{n} \bar{\Delta}_n = -\frac{1}{\sqrt{n}}
  \sum_{i = 1}^n  \frac{\varphi(Z_i)}{\psi'(0)} + o_{P,n}(1).
\end{equation*}
The conditions in
Corollary~\ref{corollary:normal-expansion}\eqref{item:apply-le-cam} imply
that the Fisher information $I_h =
\E_h[\score_h(Z)^2]$ exists and is continuous for
$\score_h(z) = \frac{p'(z - h)}{p(z - h)}$,
and the asymptotic expansion Definitions~\ref{definition:qmd}
and~\ref{definition:lan} combined
with the preceding display, give the joint convergence
\begin{equation*}
  \left(\sqrt{n} \bar{\Delta}_n, \log \frac{P^n_{h_n/\sqrt{n}}}{P^n_0}(Z_1^n)
  \right)
  \cd \normal \left( \left(0,-\frac{h^2}{2} I_0 \right),
  \begin{pmatrix}
    \frac{\chi(0)}{\psi'(0)^2} & \frac{-h}{ \psi'(0)}
    \E_p[\varphi(Z) \score_0(Z)] \\
    \frac{-h}{ \psi'(0)} \E_p[\varphi(Z) \score_0(Z)]
    & h^2 I_0
  \end{pmatrix}  \right).
\end{equation*}
Le Cam's third lemma \cite[Exm. 6.7]{VanDerVaart98} then implies the
convergence
\begin{equation*}
  \sqrt{n}\bar{\Delta}_n
  \mathop{\cd}_{P_{h_n/\sqrt{n}}^n}
  \Ncal\left(\frac{-h}{ \psi'(0)} \E_p[\varphi(Z) \score(Z)],
  \frac{\chi(0)}{\psi'(0)^2} \right)
\end{equation*}
under the alternatives $P_{h_n/\sqrt{n}}$, which gives
Corollary~\ref{corollary:normal-expansion}\eqref{item:apply-le-cam}.

\subsection{Proof of Lemma~\ref{lemma:qmd}}
\label{sec:proof-qmd}

The proof is essentially completely parallel to that of \cite[Lemma
  7.6]{VanDerVaart98}. Define $\dot{s}_\theta = \half
\frac{\dot{p}_\theta}{p_\theta} \sqrt{p_\theta}$, which exists $\mu$-almost
surely, so that $\int \dot{s}_\theta \dot{s}_\theta^\top d\mu$ is
well-defined (though it may be infinite). By Lebesgue's integration theorem,
we have
\begin{equation*}
  s_{\theta + h}(x) - s_\theta(x) = \int_0^1 h^\top \dot{s}_{\theta + t h}(x) dt,
\end{equation*}
and so Jensen's inequality (or Cauchy-Schwarz) we have
\begin{equation*}
  (s_{\theta + h}(x) - s_\theta(x))^2
  \le \int_0^1 h^\top \dot{s}_{\theta + t h}(x)
  \dot{s}_{\theta + t h}(x) ^\top h dt.
\end{equation*}
Thus for any $h_t$ we have
\begin{align*}
  \int \left(\frac{s_{\theta + t h_t}(x) - s_\theta(x)}{t}\right)^2
  d\mu(x)
  & \le \int \int_0^1 (h_t^\top \dot{s}_{\theta + u t h_t})^2 du
  d\mu \\
  & = \int_0^1 h_t^\top \int
  \dot{s}_{\theta + u t h_t}\dot{s}_{\theta + u t h_t}^\top
  d\mu(x) h_t  du
  = \frac{1}{4} h_t^\top \left(\int_0^1 I_{\theta + u t h_t} du\right) h_t.
\end{align*}
By continuity, as $h_t \to h$ and $t \to 0$ the assumed continuity
of $\theta \mapsto I_\theta$ gives that the final display converges
to $h^\top I_\theta h$.

Now, we note that
\begin{equation*}
  \lim_{t \downarrow 0}
  \left(\frac{s_{\theta + t h_t}(x) - s_\theta(x)}{t}
  - h^\top \dot{s}_\theta(x)\right)^2 = 0
\end{equation*}
for all $x$ excepting a $\mu$-null set, and the
variant dominated convergence theorem in
\cite[Prop.~2.29]{VanDerVaart98} implies that
\begin{equation*}
  \lim_{t \to 0}
  \frac{1}{t^2}
  \int \left(s_{\theta + t h_t}(x) - s_\theta(x)
  - t h^\top \dot{s}_\theta(x)\right)^2 d\mu(x)
  = \lim_{t \to 0}
  \int \left(\frac{s_{\theta + t h_t}(x) - s_\theta(x)}{t}
  - h^\top \dot{s}_\theta(x)\right)^2 d\mu(x)
  = 0,
\end{equation*}
completing the proof.
