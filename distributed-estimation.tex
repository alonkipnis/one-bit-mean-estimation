\section{Distributed Estimation \label{sec:distributed}}
In this section, we now consider the distributed encoding setting in Figure~\ref{fig:setup}-(iii) where  each one-bit message $B_i$ is only a function of its private sample $X_i$. In this case, the $i$th encoder is fully characterized its \emph{detection region}, defined as 
\[
A_i = \left\{ x \in \mathbb R \,:\, B_i(x) = 1 \right\}.
\]
Consequently, $B_i$ is of the form
\[
B_i = \begin{cases} 1 & X_i \in A_i, \\
-1 & X_i \notin A_i,
\end{cases} \quad i\in \mathbb N,
\]
where the detection region $A_i$ is a Borel set that is independent of $X_1,\ldots,X_n$.\par

As a first step, the following theorem provides conditions under which the messages $B_1,B_2,\ldots$ define a local asymptotic normal family. The proof of this theorem is in Appendix~\ref{proof:thm:LAN1}. 

\begin{thm} \label{thm:LAN1}
For $n \in \mathbb N$ and $A_n \subset \mathbb R$,  define
\begin{equation}
\label{eq:precision_general}
L_n(A_1,\ldots,A_n;\theta) \triangleq \frac{1}{n} \sum_{i=1}^n \frac{ \left(\frac{d}{d \theta} \Prob(X_i \in A_i) \right)^2 }{ \Prob(X_i \in A_i)\left(1- \Prob(X_i \in A_i) \right) }. 
\end{equation}
Consider the following conditions:
\begin{itemize}
\item [(i)\,\,] The probability density function $f(x)$ of $X_n-\theta$ is log-concave, differentiable, and symmetric. Furthermore, there exists $\delta>0$ such that the function $\eta^{1+\delta}(x)/f^\delta(x)$ is non-increasing in $|x|$. 
\item[(ii)\,] $A_n$ is a finite union of disjoint intervals.
\item[(iii)] The limit 
\begin{equation}
\label{eq:LAN_lim}
\kappa(\theta) \triangleq \lim_{n\to \infty} L_n(A_1,\ldots,A_n; \theta)
\end{equation}
exists. 
\end{itemize}
For $i=1,\ldots,n$ set
\[
B_n = \begin{cases} 1 & X_n \in A_n, \\
-1 & X_n \notin A_n. 
\end{cases} 
\]
For any $\theta$, $f(x)$ and a sequence of sets $A_1,A_2,\ldots$ such that (i)-(iii) hold, and any $h\in \mathbb R$, we have
\begin{align*}
& \log  \frac{ \mathbb P_{\theta+h/\sqrt{n}} (B_1,\ldots,B_n) }{
\mathbb P_{\theta} (B_1,\ldots,B_n)} \\
& \overset{d}{\longrightarrow} \Ncal\left(-\frac{1}{2} h^2 \kappa(\theta), h^2 \kappa(\theta) \right).
\end{align*} 
\end{thm}
Theorem~\ref{thm:LAN1} provides conditions under which $B_1,\ldots,B_n$ defines a LAN family with a precision parameter given by the limit in \eqref{eq:LAN_lim}. Condition (i) is satisfied, for example, by the normal distribution with $\delta\leq 2$. Condition (iii) is arguably the strongest and hardest to verify. As we show in Section~\ref{subsec:threshold} below, this condition is satisfied, for example, when $A_1, \ldots,A_n$ are half lines whose starting points are drawn independently from some probability measure on $\mathbb R$. Similar ideas imply that condition (iii) holds whenever we choose the intervals consisting each $A_i$ according to some pre-specified distribution. \par
%
An important conclusion of Theorem~\ref{thm:LAN1} follows from the local asymptotic minimax property of estimators in LAN models \cite{van2000asymptotic}):
\begin{cor} \label{cor:LA_minimax}
Let ${\theta}_n$ be an estimator of $\theta \in \Theta$ from $B_1,\ldots,B_n$ with detection regions $A_1,\ldots,A_n$ such that conditions (i)-(iii) of Theorem~\ref{thm:LAN1} hold. Then for any symmetric and quasi-convex function $L$, 
\begin{align*}
& \liminf_{c \to \infty} \liminf_{n \to \infty} \sup_{\tau\,:\,|\theta-\tau| \leq \frac{c}{\sqrt{n} }}  \mathbb E \left[ L\left( \sqrt{n}({\theta}_{n} - \tau) \right) \right]
\\
& \qquad \qquad \qquad \qquad \geq \mathbb E \left[ L (Z/\sqrt{\kappa(\theta)}) \right],
\end{align*}
where $Z \sim \Ncal(0,1)$. In particular, for $L(x) = x^2$,
\begin{align*}
\liminf_{c \to \infty} \liminf_{n \to \infty} \sup_{\tau\,:\,|\theta-\tau| \leq \frac{c}{\sqrt{n} }}  n \mathbb E  \left( {\theta}_{n} - \tau \right)^2 \geq 1/\kappa(\theta).
\end{align*}
\end{cor}
%
Corollary~\ref{cor:LA_minimax} says that when the sequence of one-bit messages defines a LAN model with respect to $\theta$, no estimator can attain MSE smaller than $1/\kappa(\theta)n + O(1/n)$ where $\kappa(\theta)$ is the precision parameter of the model. Consequently, the ARE  in such case is at most $\kappa(\theta)\sigma^2$. Furthermore, as we show next, no estimator in this setting can attain the optimal ARE of $\eta(0)\sigma^2$ uniformly over $ \Theta$.

\subsection{Non-existence of a Uniformly Optimal Strategy}
We now show that under LAN models, the optimal minimal risk $1/\eta(0)$ can only be attained at a finite number of points within $\Theta$. This fact implies in particular that, unlike in the adaptive setting, no distributed estimation scheme has ARE of $\eta(0)\sigma^2$ for all $\theta \in \Theta$. 

\begin{thm} \label{thm:non_existence}
Under conditions (i)-(iii) in Theorem~\ref{thm:LAN1}, assume that each $A_i$ is the union of at most $K$ intervals. The number of points $\theta \in \Theta$ satisfying $\kappa(\theta) = \eta(0)$ is at most $2K$. 
\end{thm}

\begin{proof}
See Appendix~\ref{proof:thm:non_existence}.
\end{proof}


We next consider the case where each detection region is a half-open interval, i.e., the $i$th message is obtained by comparing $X_i$ against a single threshold. As we explain next, the existence of a density for the sequence of thresholds is enough to establish local asymptotic normality and leads to a closed form expression for the precision parameter and the ARE.  

%In the adaptive setting, it follows from Theorem~\ref{thm:sgd} that messages of this kind with appropriately chosen thresholds lead to an estimator with the optimal ARE of $\eta(0)sigma^2$. However, Theorem~\ref{thm:non_existence} implies that such ARE can only be attained for a negligible subset of the parameter space. As we shall see next, messages obtained via a comparison against a sequence of thresholds cannot attain the optimal efficiency one more than a single point of $\Theta$. 

\subsection{Threshold Detection \label{subsec:threshold}}
Assume now that each $B_i$ is of the form
\begin{equation}
\label{eq:threshold_message}
B_i = \sgn(t_i - X_i) = \begin{cases} 1 & X_i< t_i, \\
-1 & X_i > t_i,
\end{cases}  
\end{equation}
where $t_i\in\mathbb R$ is the \emph{threshold} of the $i$th encoder. In other words, the detection region of $B_i$ is $A_i = (t_i,\infty)$ and $\mathbb P(X_i \in A_i) = F \left( B_i(t_i-\theta) \right)$. It follows that
\begin{equation}
L_n(A_1,\ldots,A_n;\theta) = \frac{1}{n} \sum_{i=1}^n \frac{ \left(f(t_i-\theta) \right)^2 }{F\left(t_i-\theta \right) F\left(\theta - t_i \right) }  = \frac{1}{n} \sum_{i=1}^n \eta(t_i - \theta).
\label{eq:Ln_threshold}
\end{equation}
A natural condition for the existence of the limit \eqref{eq:Ln_threshold} as $n\to \infty$ is that the empirical distribution of the threshold values converges to a probability measure. Specifically, for an interval $I \subset \mathbb R$ define
\[
\lambda_n(I) = \frac{ \card \left( I \cap \{t_1,t_2,\ldots \} \right)}{n}. 
\]
Theorem~\ref{thm:LAN1} implies:
\begin{cor} \label{cor:LAN_thresh}
Let $\{t_n\}_{n=1}^\infty$ be a sequence of threshold values such that $\lambda_n$ converges (weakly) to a probability measure $\lambda(dt)$ on $\mathbb R$. Then $\left\{ B_i = \sgn(X_i - t_i) \right\}_{i=1}^n$ is a LAN family with precision parameter
\[
\kappa(\theta) = \int_{\mathbb R} \eta(t-\theta) \lambda(dt). 
\]
\end{cor}
The condition that $\lambda_n$ converges to a probability measure is satisfied, for example, whenever the $t_1,\ldots,t_n$s are drawn independently from a probability distribution $\lambda(dt)$ on $\mathbb R$. \par
Due to local asymptotic normality of $\{B_n\}_{n=1}^\infty$, the maximum likelihood estimator (ML) of $\theta$ from $B_1,\ldots,B_n$, denoted here by 
${\theta}^{ML}_n$, is local asymptotic minimax in the sense that 
\[
\sqrt{n} \left( {\theta}^{ML}_n - \theta \right) \overset{d}{\longrightarrow} \mathcal{N} \left(0, 1/\kappa(\theta) \right). 
\]
It follows that when the density of the threshold values converges to a probability measure, the ARE of the ML estimator is $\kappa(\theta)\sigma^2$, and this ARE is maximal with respect to all local alternative estimators for $\theta$. We note that ${\theta}^{ML}_n$ is given by the root of 
\begin{equation}
\label{eq:ML}
\sum_{i=1}^n B_i \frac{f \left( t_i-\theta\right) }{F \left(B_i  (t_i-\theta)\right) },  
\end{equation}
This root is unique since the log-likelihood function is concave. Furthermore, for any $n \in \mathbb R$, we have that ${\theta}^{ML}_n  \in [t_{(1)}, t_{(n)}]$ where $t_{(i)}$ denotes the $i$th element of $\{t_1,t_2\ldots\}$. Therefore, if $\{t_1,t_2\ldots\}$ is bounded (for example $\{t_1,t_2\ldots\} \subset \Theta$), then 
\[
\lim_{n\to \infty} n \ex{\left({\theta}^{ML}_n - \theta \right)^2}  = 1/\kappa(\theta), 
\] 
so that the ML estimator attains the local asymptotic MSE of Corollary~\ref{cor:LA_minimax}. \par
%
Since $\eta(x)$ attains its maximum at the origin, we conclude that
\[
\kappa(\theta) \leq \sup_{t\in \mathbb R} \eta \left( t-\theta\right) = \eta(0).
\]
This upper bound on $\kappa(\theta)$ implies that the ARE of any distributed estimator based on a sequence of threshold detectors does not exceed $\eta(0)\sigma^2$, a fact that agrees with the lower bound under adaptive estimation derived in Theorem~\ref{thm:adpative_lower_bound}. 
%
This upper bound on $\kappa(\theta)$ is attained only when $\lambda$ is the mass distribution at $\theta$. Since $\theta$ is apriori unknown, we conclude that estimation in the distributed setting using threshold detection is strictly sub-optimal compared to the adaptive setting. In other words, the ability to choose the threshold values in an adaptive manner based on previous messages necessarily improves relative efficiency compared to a non-adaptive threshold selection.  \par

%We conclude that when the density of the threshold values converges to a probability measure, the ARE of the ML is $\kappa(\theta)$, and this ARE is maximal with respect to all local alternative estimators for $\theta$. 

%
We conclude this section by considering the density of the threshold values that maximizes the ARE $\kappa(\theta)$ under the worst choice of $\theta \in \Theta$.

\subsection{Minimax Threshold Density}
The distribution $\lambda(dt)$ that maximizes $\kappa(\theta)$, and thus minimizes $1/\kappa(\theta)$, over the worst choice of $\theta$ in $\Theta$ is given as the solution to the following optimization problem:
\begin{align}
\label{eq:var_cvx_minimax}
\begin{split}
\mathrm{maximize} \quad &  \inf_{\theta \in \Theta} \int \eta(t-\theta) \lambda(dt)
\\ 
\mathrm{subject~to} 
\quad & \lambda(dt)\geq 0,\quad \int \lambda(dt) \leq 1. 
\end{split}
\end{align}
The objective function in \eqref{eq:var_cvx_minimax} is concave in $\lambda(dt)$ and hence this problem can be solved using a convex program. We denote by $\kappa^\star$ the maximal value of \eqref{eq:var_cvx_minimax} and by $\lambda^\star(dt)$ the density that achieves this maximum. Theorem~\ref{thm:LAN1} and Corollary~\ref{cor:LAN_thresh} imply that for any $\theta \in \Theta$, the ARE of any local asymptotic minimax estimator (such as the ML estimator) from $\{B_i,\,i\in N\}$ of the form \eqref{eq:threshold_message} with $t_i \sim \lambda^\star$ i.i.d. is at least $\kappa^\star$.
%
%By discretizing the interval $[-b,b]$ using $N_\theta$ values $\theta_1,\ldots,\theta_{N_\theta}$ and the real line using $N_\lambda$ values $\lambda_1,\ldots,\lambda_{N_\lambda}$, the discrete version of \eqref{eq:var_cvx_minimax} is the following linear program (LP) in the variables $K \in \mathbb R$ and $\lambda \in \mathbb R^{N_\lambda}$:
%\begin{align}
%\label{eq:cvx_minimax}
%\begin{split}
%\mathrm{maximize} \quad &  K \\ 
%\mathrm{subject~to} 
%\quad &  K \leq \mathbf H\lambda \\
%& \lambda \geq 0,\quad   \mathbf 1^T\lambda  \leq 1,
%\end{split}
%\end{align}
%where $\mathbf H_{i,j} = \eta(t_i - \theta_j)$, $i=1,\ldots,N_\lambda$, $j = 1,\ldots,N_\theta$. 
%\begin{rem}
%The number of variables in \eqref{eq:cvx_minimax} is $N_\lambda+1$ and number of constraints is $1 + N_\lambda + N_\theta$. Since an LP has an optimal solution at which the number of constraints for which equality holds is no smaller than the number of variables \cite{papadimitriou1998combinatorial}, there exists an optimal $\lambda$ with support over no more than $N_{\theta}$ points. Therefore, in approximating the solution of \eqref{eq:var_cvx_minimax} using 
%\eqref{eq:cvx_minimax}, it is enough to take $N_\lambda = N_\theta$.
%\end{rem} 
\par
Figure~\ref{fig:minimax_support} illustrates an approximating to $\lambda^\star(dt)$ obtained by solving a discretized version of \eqref{eq:var_cvx_minimax} for the case when $f(x)$ is the normal density with variance $\sigma^2$ and $\Theta = [-T,T]$. The minimax asymptotic precision parameter $\kappa^\star$ obtained this way is illustrated in Fig.~\ref{fig:minimax_ARE} as a function of half the support size $T$. Also illustrated in these figures is $\kappa_{\unif}$ which is the precision parameter corresponding to threshold values uniformly distribution over $\Theta = [-T,T]$, namely
\begin{align}
& \kappa_{\unif} \triangleq \min_{\theta \in [-T,T]} \frac{1}{2T}\int_{-T}^T \eta\left(t-\theta\right) dt \nonumber
 \\
& = 
\frac{1}{2T}\int_{-T}^{T} \eta\left(t\pm T\right) dt
= \frac{1}{2T}\int_{0}^{2T} \eta(t) dt  \label{eq:uniform_risk}. 
\end{align}
Corollary~\ref{cor:LAN_thresh} implies that the ARE under threshold values uniformly distributed is $\kappa_\unif \sigma^2$. \par

\begin{figure}
  %%%%% COMMENTED TIKZ PICTURE %%%%%
  \begin{center}
%% \begin{tikzpicture}[scale = 0.55]
%% \begin{axis}[
%% width=8cm, height=6cm,
%% xmin = -0.5, xmax=0.5, 
%% restrict y to domain = 0:100,
%% ymin = 0,
%% ymax = 1,
%% samples=10, 
%% xlabel= {$dt$},
%% xtick={-0.5,0,0.5},
%% xticklabels={-$T$, 0, $T$},
%% title = {$\sigma/T = 1/2$},
%% ytick={0,0.1,1},
%% %ylabel = {$\lambda(dt)$},
%% yticklabels={0,0.1,1},
%% line width=1.0pt,
%% mark size=1.5pt,
%% ymajorgrids,
%% xmajorgrids,
%% legend style= {at={(1,1)},anchor=north east,draw=black,fill=white,align=left}
%% ]
%% \addplot[color = blue, smooth, mark = o ] plot table [col sep=comma] {./Figs/minmax_lmd_b0.5_sig0.5.csv};

%% \addplot[color = black!30!green, smooth, dashed] plot table [x = x, y  = z,col sep=comma] {./Figs/minimax_th_b0.5_sig0.5.csv};

%% \addplot[color = red, smooth] plot table [x = x, y  = y,col sep=comma] {./Figs/minimax_th_b0.5_sig0.5.csv};

%% \end{axis}
%% \end{tikzpicture}
%% %
%% \begin{tikzpicture}[scale = 0.55]
%% \begin{axis}[
%% width=8cm, height=6cm,
%% xmin = -0.5, xmax=0.5, 
%% restrict y to domain = 0:100,
%% ymin = 0,
%% ymax = 0.6,
%% samples=10, 
%% xlabel= {$dt$},
%% title = {$\sigma/T = 1/5$},
%% xtick={-0.5,0,0.5},
%% xticklabels={-$T$, 0, $T$},
%% ytick={0,0.5},
%% ylabel = {$\lambda(dt)$},
%% yticklabels={0,0.5},
%% line width=1.0pt,
%% mark size=1.5pt,
%% ymajorgrids,
%% xmajorgrids,
%% legend style= {at={(1,1)},anchor=north east,draw=black,fill=white,align=left}
%% ]
%% \addplot[color = blue, smooth, mark = o ] plot table [col sep=comma] {./Figs/minmax_lmd_b0.5_sig0.2.csv};

%% \addplot[color = black!30!green, smooth, dashed] plot table [x = x, y  = z,col sep=comma] {./Figs/minimax_th_b0.5_sig0.2.csv};

%% \addplot[color = red, smooth] plot table [x = x, y  = y,col sep=comma] {./Figs/minimax_th_b0.5_sig0.2.csv};

%% \end{axis}
%% \end{tikzpicture}
%% %
%% \begin{tikzpicture}[scale = 0.55]
%% \begin{axis}[
%% width=8cm, height=6cm,
%% xmin = -0.5, xmax=0.5, 
%% restrict y to domain = 0:100,
%% ymin = 0,
%% ymax = 0.3,
%% samples=10, 
%% xlabel= {$dt$},
%% xtick={-0.5,0,0.5},
%% title = {$\sigma / T = 1/10$},
%% xticklabels={-$T$, 0, $T$},
%% ytick={0,0.2},
%% %ylabel = {$\lambda(dt)$},
%% yticklabels={0,0.2},
%% line width=1.0pt,
%% mark size=1.5pt,
%% ymajorgrids,
%% xmajorgrids,
%% legend style= {at={(1,1)},anchor=north east,draw=black,fill=white,align=left}
%% ]
%% \addplot[color = blue, smooth, mark = o ] plot table [col sep=comma] {./Figs/minmax_lmd_b0.5_sig0.1.csv};

%% \addplot[color = black!30!green, smooth, dashed] plot table [x = x, y  = z,col sep=comma] {./Figs/minimax_th_b0.5_sig0.1.csv};

%% \addplot[color = red, smooth] plot table [x = x, y  = y,col sep=comma] {./Figs/minimax_th_b0.5_sig0.1.csv};

%% \end{axis}
%% \end{tikzpicture}
%% %
%% \begin{tikzpicture}[scale = 0.55]
%% \begin{axis}[
%% width=8cm, height=6cm,
%% xmin = -0.5, xmax=0.5, 
%% restrict y to domain = 0:100,
%% ymin = 0,
%% ymax = 0.15,
%% samples=10, 
%% xlabel= {$dt$},
%% title = {$\sigma / T = 1/20$},
%% xtick={-0.5,0,0.5},
%% xticklabels={-$T$, 0, $T$},
%% ytick={0,0.1},
%% ylabel = {$\lambda(dt)$},
%% yticklabels={0,0.1},
%% line width=1.0pt,
%% mark size=1.5pt,
%% ymajorgrids,
%% xmajorgrids,
%% legend style= {at={(1,1)},anchor=north east,draw=black,fill=white,align=left}
%% ]
%% \addplot[color = blue, smooth, mark = o ] plot table [x = x, y  = y,col sep=comma] {./Figs/minmax_lmd_b0.5_sig0.05.csv};

%% \addplot[color = black!30!green, smooth, dashed] plot table [x = x, y  = z,col sep=comma] {./Figs/minimax_th_b0.5_sig0.05.csv};

%% \addplot[color = red, smooth] plot table [x = x, y  = y,col sep=comma] {./Figs/minimax_th_b0.5_sig0.05.csv};


%% %\addlegendentry{$\frac{\sigma^2}{\sigma_\theta^2} = 1$};
%% \end{axis}
%% \end{tikzpicture}

\caption{\label{fig:minimax_support}
Optimal threshold density $\lambda^\star(dt)$ (blue) that maximizes the ARE for $f(x) = \Ncal(\theta,\sigma^2)$ and $\theta \in [-T,T]$.
%
The continuous curve (red) represents the reciprocal of the asymptotic risk at a fixed $\theta \in [-T,T]$ under the optimal density, i.e., the minimax risk is the inverse of the minimal value of this curve. The dashed curve (green) is the reciprocal of the asymptotic risk at a fixed $\theta \in [-T,T]$ when the threshold values are uniformly distributed over $[-T,T]$, hence its minimal value is the inverse of \eqref{eq:uniform_risk}. }
\end{center}
\end{figure}


\begin{figure}
  %%%%% COMMENTED TIKZ PICTURE %%%%%
  \begin{center}
\begin{tikzpicture}[scale = 1]
\begin{axis}[
width=8cm, height=6cm,
xmax=2.5, 
xmode = log,
restrict y to domain = 0:100,
ymin = 0,
ymax = 1,
samples=1, 
xlabel= {$\sigma/T$},
%xtick={-0.5,0,0.5},
%xticklabels={-$$, 0, $b$},
ytick={0,0.637,1},
yticklabels={0,$2/\pi$,1},
ylabel = {$\ARE$},
line width=1.0pt,
mark size=1.5pt,
ymajorgrids,
xmajorgrids,
legend style= {at={(1,1)},anchor=north east,draw=black,fill=white,align=left}
]
\addplot[color = red, smooth, line width = 1pt] plot table [x = x, y = y, col sep=comma] {./Figs/minmax_ARE_b0.5.csv};
\addlegendentry{optimal threshold density};

\addplot[color = black!35!green, smooth, dashed, line width = 1pt] plot table [x = x, y = z, col sep=comma] {./Figs/minmax_ARE_b0.5.csv};
\addlegendentry{uniform threshold desnity};
\end{axis}
\end{tikzpicture}
\caption{\label{fig:minimax_ARE} Minimax ARE versus $\sigma/T$ for $f(x) = \Ncal(\theta,\sigma^2)$  and $\theta \in \Theta = [-T,T]$. The dashed curve (green) is the ARE under a uniform threshold density over $\Theta$ given by $K_{\unif}\sigma^2$, where $\kappa_{\unif}$ is given by \eqref{eq:uniform_risk}. }.
\end{center}
\end{figure}


%We have seen that using threshold detection, an ARE $\eta(0)\sigma^2$ can only be attained at a single point in $\Theta$. Next, we prove that with any set of messages obtained via a distributed strategy, there exists at least one $\theta \in \Theta$ such that the estimator of $\theta$ has ARE strictly smaller than $\eta(0)\sigma^2$. 

\begin{figure}
\begin{center}
%
\begin{tikzpicture}[scale = 0.55]
\begin{axis}[
width=8cm, height=6cm,
xmin = -0.5, xmax=0.5, 
restrict y to domain = 0:100,
ymin = 0,
ymax = 1,
samples=10, 
xlabel= {$dt$},
title = {$\sigma/\sigma_{\theta} = 4$},
xtick={-0.5,0,0.5},
xticklabels={-$T$, 0, $T$},
ytick={0,1},
ylabel = {$\lambda(dt)$},
yticklabels={0,1},
line width=1.0pt,
mark size=1.5pt,
ymajorgrids,
xmajorgrids,
legend style= {at={(1,1)},anchor=north east,draw=black,fill=white,align=left}
]
\addplot[color = blue, smooth, mark = o ] plot table [col sep=comma] {./Figs/unif_Bayes_lmd_b0.5_sig4.csv};
%\addlegendentry{$\frac{\sigma^2}{\sigma_\theta^2} = 1$};
\end{axis}
\end{tikzpicture}
%
\begin{tikzpicture}[scale = 0.55]
\begin{axis}[
width=8cm, height=6cm,
xmin = -0.5, xmax=0.5, 
restrict y to domain = 0:100,
ymin = 0,
ymax = 1,
samples=10, 
xlabel= {$dt$},
xtick={-0.5,0,0.5},
title = {$\sigma/\sigma_{\theta} = 3$},
xticklabels={-$T$, 0, $T$},
ytick={0,1},
%ylabel = {$\lambda(dt)$},
yticklabels={0,1},
line width=1.0pt,
mark size=1.5pt,
ymajorgrids,
xmajorgrids,
legend style= {at={(1,1)},anchor=north east,draw=black,fill=white,align=left}
]
\addplot[color = blue, smooth, mark = o ] plot table [col sep=comma] {./Figs/unif_Bayes_lmd_b0.5_sig3.csv};
%\addlegendentry{$\frac{\sigma^2}{\sigma_\theta^2} = 1$};
\end{axis}
\end{tikzpicture}
%
\begin{tikzpicture}[scale = 0.55]
\begin{axis}[
width=8cm, height=6cm,
xmin = -0.5, xmax=0.5, 
restrict y to domain = 0:100,
ymin = 0,
ymax = 1,
samples=10, 
xlabel= {$dt$},
title = {$\sigma/\sigma_{\theta} = 2$},
xtick={-0.5,0,0.5},
xticklabels={-$T$, 0, $T$},
ytick={0,1},
ylabel = {$\lambda(dt)$},
yticklabels={0,1},
line width=1.0pt,
mark size=1.5pt,
ymajorgrids,
xmajorgrids,
legend style= {at={(1,1)},anchor=north east,draw=black,fill=white,align=left}
]
\addplot[color = blue, smooth, mark = o ] plot table [col sep=comma] {./Figs/unif_Bayes_lmd_b0.5_sig2.csv};
%\addlegendentry{$\frac{\sigma^2}{\sigma_\theta^2} = 1$};
\end{axis}
\end{tikzpicture}
%
\begin{tikzpicture}[scale = 0.55]
\begin{axis}[
width=8cm, height=6cm,
xmin = -0.5, xmax=0.5, 
restrict y to domain = 0:100,
ymin = 0,
ymax = 1,
samples=10, 
xlabel= {$dt$},
xtick={-0.5,0,0.5},
xticklabels={-$T$, 0, $T$},
title = {$\sigma/\sigma_{\theta} = 1$},
ytick={0,1},
%ylabel = {$\lambda(dt)$},
yticklabels={0,1},
line width=1.0pt,
mark size=1.5pt,
ymajorgrids,
xmajorgrids,
legend style= {at={(1,1)},anchor=north east,draw=black,fill=white,align=left}
]
\addplot[color = blue, smooth, mark = o ] plot table [col sep=comma] {./Figs/unif_Bayes_lmd_b0.5_sig1.csv};
%\addlegendentry{$\frac{\sigma^2}{\sigma_\theta^2} = 1$};
\end{axis}
\end{tikzpicture}
\caption{\label{fig:opt_density}
Optimal threshold density $\lambda^\star(dt)$ that minimizes the asymptotic Bayes risk \eqref{eq:cvx_average} for a uniform prior with $\sigma/\sigma_\theta=1,2,3,4$, where $\sigma_\theta^2=T^2/3$ is the variance of the prior. 
}
\end{center}
\end{figure}

When a prior $\pi$ on $\Theta$ is provided, one may be interested in the threshold density $\lambda$ minimizing the Bayes risk $R_n(\pi)$. The resulting minimization problem is 
\begin{align}
\label{eq:cvx_average}
\begin{split}
\mathrm{minimize} \quad & R_{\pi} =  \int \frac{\pi(d\theta)}{ \int \eta \left( t-\theta\right) \lambda(dt)}. \\ 
\mathrm{subject~to} \quad & \lambda(dt)\geq 0,\quad \int \lambda(dt) =1, 
\end{split}
\end{align}
which is convex in $\lambda$ since $x \rightarrow 1/x$, $x>0$, is convex. Figure~\ref{fig:opt_density}  illustrates the solution to \eqref{eq:cvx_average} for the case of where $f(x)$ is the normal distribution and $\pi$ is the uniform over $\Theta = [-T,T]$. %Figure~\ref{fig:dist_bound_uniform} illustrates the corresponding Bayes risk. 

\iffalse
\begin{figure}
\begin{center}
\begin{tikzpicture}[scale = 1]
\begin{axis}[
width=8cm, height=6cm,
xmin = 0, xmax=2, 
restrict y to domain = 0:100,
ymin = 0,
ymax = 10,
samples=1, 
xlabel= {$\sigma/\sigma_\theta$},
%xtick={-0.5,0,0.5},
%xticklabels={-$$, 0, $b$},
ytick={0,1.5708,5},
yticklabels={0,$\pi/2$,5},
ylabel = {$R_{\pi}^\star/\sigma^2$},
line width=1.0pt,
mark size=1.5pt,
ymajorgrids,
xmajorgrids,
legend style= {at={(1,1)},anchor=north east,draw=black,fill=white,align=left}
]
\addplot[color = red, smooth] plot table [x = x, y = Runif, col sep=comma] {./Figs/unif_Bayes_Risk.csv};
%\addlegendentry{$\frac{\sigma^2}{\sigma_\theta^2} = 1$};
\addplot[color = red, smooth, dashed, line width = 0.5pt] plot table [x = x, y = Rbound, col sep=comma] {./Figs/unif_Bayes_Risk.csv};
\addplot[color = black, dashed] {3.14159 / 2};
\end{axis}
\end{tikzpicture}
\caption{Asymptotic Bayes risk $R_{\pi}^\star$ in estimating the mean of a normal distribution ($P_X = \Ncal(\theta, \sigma^2)$ under an optimal threshold distribution $\lambda^\star$, when the prior $\pi$ is the uniform distribution over $\Theta = [-0.5,0.5]$. 
%
%versus $\sigma/\sigma_\theta = 2\sqrt{3}\sigma$.
The distribution $\lambda^\star$ is the minimizer of \eqref{eq:cvx_average}. It is illustrated for various cases in Fig.~\ref{fig:opt_density}. The dashed curve represents the upper bound \eqref{eq:upper_bound}. 
\label{fig:dist_bound_uniform}  }
\end{center}
\end{figure}

As can be seen in Fig.~\ref{fig:opt_density} for the case $P_X = \Ncal(\theta ,\sigma^2)$ and a uniform $\pi$, when the radius of $\Theta$ is small compared to $\sigma$, the optimal distribution $\lambda^\star$ is a mass distribution. In this case, the ML estimator reduces to the estimator ${\theta}_n$ of \eqref{eq:estimator_naive}. As the following proposition shows, the Bayes risk for this choice of $\lambda$ is maximal, and thus provides an upper bound on the Bayes risk under any $\lambda$. 
 \eqref{eq:cvx_average}. 
\begin{prop}\label{prop:upper_bound}
For any prior $\pi(d\theta)$ and $\theta_0$ in the support of $\eta(x)$ we have
\begin{equation} 
\label{eq:upper_bound}
R_\pi^\star  \leq 
 \int \frac{\pi(d\theta)}{\eta \left( \theta_0 - \theta \right)},
\end{equation}
Furthermore, assuming that $\theta_0 = \ex{\theta}$ and that $\pi$ has a finite third moment $\sigma_\theta^3$, we have: 
\begin{equation}
\label{eq:bound_Taylor}
R^\star_\pi \leq \frac{1}{4 f^2(0)} + \left(\frac{1}{4 f^2(0)} \frac{-f''(0)}{f(0)} -1 \right) \sigma_\theta^2 + O(\sigma_\theta^3).
\end{equation}
\end{prop}
\begin{proof}
The function $x \rightarrow 1/x$ is convex for positive values, hence Jensen's inequality implies
\[
\left( \int \eta \left( t-\theta\right) \lambda(dt) \right)^{-1}  \leq \int  \frac{ \lambda(dt)}{ \eta \left( t-\theta\right)  }. 
\]
Therefore, the expected value of $\kappa^{-1}(\theta)$ satisfies
\begin{align}
\ex{  \frac{1}{\kappa(\theta)}}  \leq \int \int \frac{\pi(d\theta) \lambda(dt) }{\eta \left( t - \theta \right)}. \label{eq:upper_bound_proof}
\end{align}
The bound \eqref{eq:upper_bound} is obtained by taking $\lambda$ to be a mass distribution at any $\theta_0$ in the support of $\eta(x)$. Finally, \eqref{eq:bound_Taylor} is obtained by expanding $1/\eta(x)$ to a third order Taylor series around zero
%\[
%1/\eta(x) = \frac{1}{4f^2(0)}  + \left(\frac{1}{4 f^2(0)} \frac{-f''(0)}{f(0)} -1 \right)  x^2 + O(x^3),
%\]
and taking its expectation with respect to $\pi$ at $x=\theta_0-\theta$. 
\end{proof}
We note that the function $1/\eta(x)$ is quasi-convex and symmetric around zero, so the choice $\theta_0 = \ex{ \theta}$ minimizes the RHS of \eqref{eq:upper_bound} among all $\theta_0$ in the support of $\eta(x)$.\par
The bound \eqref{eq:upper_bound} is not trivial as long as the integral in the RHS of \eqref{eq:upper_bound} is finite, i.e., whenever the tail of $\pi(\theta)$ vanishes fast enough compared to $\eta(x)$. The expansion \eqref{eq:bound_Taylor} implies that this bound becomes tight whenever the support of the optimal distribution is a mass distribution at $\mathbb E \theta$, in which case the expected value of $\kappa^{-1}(\theta)$ approaches $1/\eta(0) = 1/4f^2(0)$. 

\begin{example} \label{ex:bound}
In the normal case $P_X = \Ncal(\theta, \sigma^2)$, the bound in \eqref{eq:bound_Taylor} implies 
\[
\frac{R^\star_\pi}{\sigma^2}  \leq \frac{\pi}{2} + \left( \frac{\pi}{2} -1 \right) \left( \frac{ \sigma_\theta}{\sigma} \right)^2 + O \left(  \frac{\sigma_\theta} { \sigma} \right)^3. 
\]
It follows that the ARE approaches its maximal value of $2/\pi$ whenever $\sigma_\theta/\sigma$ is small. The exact value of \eqref{eq:upper_bound} in this case, as well as the Bayes ARE with the optimal threshold density $\lambda^\star$ for a uniform $\pi$, are illustrated in Fig.~\ref{fig:dist_bound_uniform}. 
%i.e., each encoder simply reports "$1$" or "$-1$" whenever $X_i$ is larger or smaller than $\theta_0$, respectively. Intuitively, an accurate estimation of $\theta$ in this case is possible only if a sufficient mix of ''$1$''s and ''$-1$''s is obtained from the sample. When $\sigma_\theta^2$ is high compared to $\simga^2$, one of the events $"1"$ or "$-1"$ becomes too rare to
\end{example}
\fi
