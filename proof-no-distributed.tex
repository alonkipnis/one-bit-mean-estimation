% -*- Mode: latex -*- %

\section{Proof of Theorem~\ref{thm:non_existence}}
\label{proof:thm:non_existence}

Let $\Xi$ be the set of points $\theta \in \Theta$ for which $\kappa(\theta) = \eta(0)$. 
%
Since $B_1,B_2,\ldots$ satisfy the conditions in Theorem~\ref{thm:LAN1}, $\theta$ is in $\Xi$ if and only if
\begin{equation}
\label{eq:non_existence_proof}
\lim_{n\to \infty} L_n(A_1,\ldots,A_n;\theta) = \eta(0). 
\end{equation}
By assumption, we have $B_i^{-1} = A_i$ where $A_i$ can be expressed as
\[
A_i = \cup_{i=1}^K (a_{i,k},b_{i,k}), 
\]
where $a_{i,1} \leq b_{i,1} \leq \ldots \leq a_{i,K}, b_{i,K}$, and $a_{i,1}$ and $b_{i,K}$ may take the values $-\infty$ and $\infty$, respectively. Denote
\[
C_i = \cup_{k=1}^{K}\{a_{i,k},b_{i,k}\}.
\]
For any $\theta$ and $\epsilon>0$, denote 
\[
S_n(\theta, \epsilon) \triangleq \left\{ i\leq n \,:\, (\theta-\epsilon,\theta+\epsilon) \cap C_i \neq \emptyset \right\}
\]
In words, $S_n$ contains all integers smaller than $n$ in which an $\epsilon$-ball around $\theta$ contains an endpoint of one of the intervals consisting $A_i$. 
%
We now claim that 
if $\theta \in \Xi$ then $\card(S_n(\theta, \epsilon))/n \to 1$. Indeed, for such $\theta$ we have
\begin{align}
& L_n(A_1,\ldots,A_n; \theta) \nonumber \\
& = \frac{1}{n} \sum_{i \in S_n(\epsilon,\theta)}  
\frac{ \left(\sum_{k=1}^{K}  f(\theta - b_{i,k})- f(\theta - a_{i,k}) \right)^2}{ \sum_{k=1}^{K} \left( F(\theta - b_{i,k})- F(\theta - a_{i,k}) \right) \left(1-\sum_{k=1}^{K} \left( F(\theta - b_{i,k})- F(\theta - a_{i,k}) \right)\right)} \nonumber \\
& 
+ \frac{1}{n}\sum_{i \notin S_n(\epsilon,\theta) } \frac{ \left(\sum_{k=1}^{K}  f(b_{i,k}-\theta) - f(a_{i,k}-\theta) \right)^2} { \sum_{k=1}^{K} \left( F(\theta - b_{i,k})- F(\theta - a_{i,k}) \right) \left(1-\sum_{k=1}^{K} \left( F(\theta - b_{i,k})- F(\theta - a_{i,k}) \right)\right)} \nonumber \\
& \overset{(a)}{\leq}
\frac{\card\left(S_n(\theta,\epsilon)\right)}{n} \eta(0) + \frac{n-\card\left(S_n(\theta,\epsilon) \right) }{n} \eta(\epsilon) 
 \label{eq:non_existence_proof1}
\end{align}
where $(a)$ follows from Lemma~\ref{lem:bound_intervals_delta} with $\delta =0$, and the fact that for $i \in S_n(\theta, \epsilon)$, 
\[
\max\left\{ \max_k \eta(b_{i,k}-\theta) , \max_k \eta(a_{i,k}-\theta)  \right\} \leq \eta(\epsilon) < \eta(0). 
\]
Unless  $\card \left(S_n(\theta, \epsilon) \right)/n \to 1$, we get that \eqref{eq:non_existence_proof1}, hence $L_n(A_1,\ldots,A_n ; \theta)$, are bounded from above by a constant that is smaller then $\eta(0)$ in contradiction to the fact that $\theta \in \Xi$. \par
For $k\in \N$, assume by contradiction that there exists $N \geq 2K + 1$ distinct elements
$\theta_1,\ldots,\theta_N \in \Xi$. Since each $A_i$ consists of at most $K$ intervals, we have that 
\begin{equation}
\label{eq:few_optimality_points_proof}
\card (\cup_{i=1}^n \mathcal B_i) \leq 2 n K. 
\end{equation}
Fix $\epsilon>0$ such that 
\[
\epsilon < \frac{1}{2}\min_{i\neq j} |\theta_i - \theta_j|. 
\]
Since for each $\theta \in \Theta$ we have $S_n(\theta, \epsilon) \to 1$, there exists $n$ large enough such that 
\[
\card \left(S_n(\theta_i, \epsilon) \right) \geq n \left(1-\frac{1}{2N} \right)
\]
for all $i=1,\ldots,N$. However, $S_n(\theta_1,\epsilon), \ldots S_n(\theta_N,\epsilon)$ are disjoint, so the cardinallity of their union is at least $n\left(1-\frac{1}{2N} \right)N$ which is grater than $2nK + n/2$ in contradiction to \eqref{eq:few_optimality_points_proof}. 

