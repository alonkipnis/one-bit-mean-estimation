% -*- Mode: latex -*- %

\section{Problem Formulation \label{sec:problem}}

Let $f(x)$ be a symmetric and log-concave density function with a finite second moment $\sigma^2$. For $\theta \in \Theta$, denote by $P_X$ the probability distribution with density $f\left( x-\theta \right)$. Therefore, $P_X$ is an absolutely continuous log-concave distribution with mean $\theta$ and variance $\sigma^2$. Symmetry and log-concavity of $f(x)$ imply that $P_X$ is strongly unimodal (has a unique local maxima) with its mode at $x =\theta$ \cite{ibragimov1956composition}. We further assume that the \emph{parameter space} $\Theta$ is a closed interval of the real line. 
\par
In some situations it is useful to assume that $\theta$ is drawn once from the prior distribution $\pi$ on $\Theta$. In this case we assume that $\pi$ is an absolutely continuous distribution, and denote its density by $\pi(\theta)$, i.e., $\pi(d\theta) = \pi(\theta)d\theta$.  
\par
The random variables $X_1,\ldots,X_n$ represent $n$ independent samples from $P_X$. 
We are interested in estimating $\theta$ from a set of $n$ binary messages $B_1,\ldots,B_n$, obtained from $X_1,\ldots,X_n$ under three possible scenarios illustrated in Figure~\ref{fig:setup}: 
\begin{itemize}
\item[(i)~~] Centralized $B_i(X_1,\ldots,X_n)$, $i=1,\ldots,n$.
\item[(ii)~] Adaptive $B_i(X_i,B_1,\ldots,B_{i-1})$, $i=2,\ldots,n$.
\item[(iii)] Distributed $B_i(X_i)$, $i=1,\ldots,n$.
\end{itemize}

The performance of an estimator ${\theta}_n \triangleq {\theta}_n(B^n)$ in any of these cases is measured according to the quadratic risk:
\begin{equation}
\label{eq:error_def}
R_n \triangleq \mathbb E\left({\theta}_n - \theta \right)^2,
\end{equation}
where the expectation is taken with respect to the distribution of $X_1,\ldots,X_n$ and, whenever available, a prior distribution $\pi(\theta)$ over $\Theta$.  
%It is well known that minmax estimation error 
%\begin{equation}
%R_n = \min \sup_{\theta \in \Theta}  \mathbb E\left[ \left({\theta}_n - \theta \right)^2 | \theta \right],
%\end{equation}
%can be obtained from $R_n$ by assuming a particular least favorable prior distribution on the parameter space $\Theta$ \cite{casella1981}. 
%
The main problems we consider in this paper are the minimal value of \eqref{eq:error_def}, as a function of $n$ and $f(x)$, under different choices of the encoding functions in cases (i), (ii), and (iii). \par

We give particular attention to the ARE of estimators with respect to an asymptotically normal efficient estimator that is not subject to the bit constraint. Specifically, let $\{a_n,\,n\in \mathbb N\}$ be a sequence such that 
\[
\sqrt{a_n}\left({\theta}_n - \theta\right) \overset{d}{\longrightarrow} \Ncal(0, \sigma^2).
\]
Then the ARE of ${\theta}_n$ with respect to an unconstrained efficient estimator for $\theta$ is defined as \cite[Def. 6.6.6]{lehmann2006theory}
\[
\ARE({\theta}_n) \triangleq
\lim_{n\rightarrow \infty} \frac{a_n}{n}. 
\]
Note that in the special case where there exists $V \in \mathbb R$ such that
\[
a_n \mathbb E \left({\theta}_n - \theta \right)^2 = V + o(1),
\]
with $o(1)\to 0$ as $n\to \infty$, then the ARE of ${\theta}_n$ is finite and equals $\sigma^2/V$. \par

In addition to the notation above, we also denote by $F(x)$ the cumulative distribution function of $X_i$, and define
\begin{equation} \label{eq:eta_def}
\eta(x) \triangleq \frac{f^2(x)}{F(x)(1-F(x))} =  \frac{f(x)f(-x)}{F(x)F(-x)}, 
\end{equation}
where the last equality is due to symmetry of $f(x)$. We note that 
\begin{equation}
\label{eq:eta_h}
\eta(x) = h(x)h(-x) = f(x) \left( h(x) + h(-x) \right), 
\end{equation}
where 
\[
h(x) \triangleq \frac{f(x)}{1-F(x)} = \frac{f(x)}{F(-x)}
\]
is the \emph{hazard} function (a.k.a. \emph{failure rate} or \emph{force of mortality}), which is a monotone increasing function since $f(x)$ is log-concave \cite{bagnoli2005log}. 
%
For $f(x)$ the normal density, it is shown in \cite{Samford1953} and \cite{hammersley1950estimating} that $\eta(x)$ is a strictly decreasing function of $|x|$, as illustrated in Fig.~\ref{fig:eta}. 
%
In this paper we only consider the normal distribution and other log-concave symmetric distributions for which this property of $\eta(x)$ holds. Specifically, we require the following:
\begin{assump} \label{assump:failure_rate}
 $\eta(x)$ has a unique maximum (at the origin) and is a non-increasing function of $|x|$. 
\end{assump}
%That is, Assumption~\ref{assump:failure_rate} says that $\eta'(x) < 0 $ for all $x>0$. 
Under this assumption we have, 
\[
4f^2(x) \leq \eta(x) \leq \eta(0),
\] 
%
where $\eta(0) = 4 f^2(0)$ is the asymptotic variance of the sample median. Combined with log-concavity of $f(x)$, Assumption~\ref{assump:failure_rate} implies that $\eta(x)$ vanishes as $|x|\rightarrow \infty$. \par
%
Assumption~\ref{assump:failure_rate} is satisfied, for example, by the generalized normal distributions with a shape parameter between $1$ and $2$ (including normal and Laplace distributions). Symmetric log-concave distributions that do not satisfy Assumption~\ref{assump:failure_rate} include the uniform distribution and the generalized normal distribution with shape parameter greater than $2$.\par

%% \begin{figure}
%% \begin{center}
%% \begin{tikzpicture}[scale = 0.6]
%% \begin{axis}[
%% width=8cm, height=6cm,
%% xmin = -3, xmax=3, 
%% restrict y to domain = 0:100,
%% ymin = 0,
%% ymax = 0.9,
%% samples=10, 
%% xlabel= $x$,
%% xtick={-2,-1,0,1,2},
%% xticklabels={-2,-1,0,1,2},
%% ytick={0,0.3989423,0.6366198},
%% yticklabels={0,$\frac{1}{\sqrt{2 \pi}}$,$2/\pi$},
%% line width=1.0pt,
%% mark size=1.5pt,
%% ymajorgrids,
%% xmajorgrids,
%% legend style= {at={(1,1)},anchor=north east,draw=black,fill=white,align=left}
%% ]
%% \addplot[color = blue, solid, smooth] plot table [x = x, y = y, col sep=comma] {./Figs/eta.csv};
%% \addlegendentry{$\eta(x)$};
%% \addplot[domain = -5:5, samples = 50, color = red, solid, smooth]  {exp(-x^2/2) / sqrt(2*3.14159)};
%% \addlegendentry{$\phi(x)$};
%% \addplot[domain = -5:5, samples = 50, color = black, solid, dashed]  {4*exp(-x^2) / (2*3.14159)};
%% \addlegendentry{$4\phi^2(x)$};

%% \end{axis}
%% \end{tikzpicture}
%% %
%% \begin{tikzpicture}[scale = 0.6]
%% \begin{axis}[
%% width=8cm, height=6cm,
%% xmin = -4, xmax=4, 
%% restrict y to domain = -10:0,
%% ymin = -10,
%% samples=10, 
%% xlabel= $x$,
%% xtick={-3,-2,-1,0,1,2,3},
%% xticklabels={-3,-2,-1,0,1,2,3},
%% ytick={0,-0.45158,-0.919},
%% yticklabels={0,,},
%% line width=1.0pt,
%% mark size=1.5pt,
%% ymajorgrids,
%% xmajorgrids,
%% legend style= {at={(1,1)},anchor=north east,draw=black,fill=white,align=left}
%% ]
%% \addplot[color = blue, solid, smooth] plot table [x = x, y = logy, col sep=comma] {./Figs/eta.csv};
%% \addlegendentry{$\log \eta(x)$};
%% \addplot[domain = -5:5, samples = 30, color = red, solid, smooth]  {-(x)^2/2 -0.9189};
%% \addlegendentry{$\log \phi(x)$};
%% \end{axis}
%% \end{tikzpicture}
%% \caption{
%% The function $\eta(x) = f^2(x) / F(x)F(-x)$ for $f(x) = \phi(x)$ the standard normal density.
%% \label{fig:eta}
%% }
%% \end{center}
%% \end{figure}
