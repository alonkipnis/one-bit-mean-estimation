% -*- Mode: latex -*- %

\section{Problem Formulation and Notation}
\label{sec:problem}

Let $f : \R \to \R_+$ be a symmetric and log-concave probability density,
which necessarily has finite second moment $\sigma^2$, and let $\Theta
\subset \R$ be closed and convex.  For $\theta \in \R$, let $P_\theta$ be
the probability distribution with density $f(x-\theta)$, so that $\theta$
indexes the location family $\{P_\theta\}_{\theta \in \Theta}$.  The
log-concavity and symmetry $f(x)$ imply that $P_\theta$ has a unique mean
and median at $\theta$~\cite{ibragimov1956composition}.
We observe a sample $X_1, \ldots, X_n \simiid P_\theta$, where $\theta$
is unknown, and wish to estimate $\theta$ given only binary
messages $B_1, \ldots, B_n \in \{0, 1\}$ about each $X_i$.
We study this under three distinct computational scenarios, which
we illustrate in Figure~\ref{fig:setup}:
\begin{enumerate}[(i)]
\item \label{item:centralized} Centralized, where $B_i =
  B_i(X_1,\ldots,X_n)$, $i=1,\ldots,n$.
\item \label{item:adaptive} Adaptive, where $B_i =
  B_i(X_i,B_1,\ldots,B_{i-1})$, $i=2,\ldots,n$.
\item \label{item:distributed}
  Distributed, where $B_i = B_i(X_i)$, $i=1,\ldots,n$.
\end{enumerate}
\noindent
We also consider a hybrid of the fully distributed setting (where the bits
$B_i$ are independent) and the adaptive setting (where each bit $B_i$ may
depend on the previous bits) to a \emph{one-step} adaptive setting, where
the quantization scheme may be modified to depend on one fixed function of
the previous information.
\begin{enumerate}[(i')]
\setcounter{enumi}{1}
\item \label{item:one-step-adaptive}
  One-step adaptive, where for some function $g$ and
  a (fixed) $t$, if $i \le t$ then
  $B_i = B_i(X_i)$ while if $i > t$, then
  $B_i = B_i(X_i, g(B_1, \ldots, B_t))$.
\end{enumerate}

We measure the performance of an estimator $\theta_n \defeq
\theta_n(B_1,\ldots,B_n)$ by one of a few notions. In the simplest case,
we assume a prior $\pi$ on $\theta$ (which may be a point
mass) and consider the quadratic risk
\begin{equation}
  \label{eq:error_def}
  R_n = R_n(\pi) \defeq \int \E_\theta\left({\theta}_n - \theta \right)^2
  d\pi(\theta),
\end{equation}
where the expectation is taken with respect to the distribution of
$X_1,\ldots,X_n \simiid P_\theta$.  The main problems we consider in this
paper are the minimal value of the risk~\eqref{eq:error_def} as a function
of the sample size $n$ and the density $f$, under different choices of the
encoding functions in
cases~\eqref{item:centralized}--\eqref{item:distributed}.
The quadratic risk~\eqref{eq:error_def} may be infinite in some cases;
we defer discussion of this case to later sections, as it is technically
demanding and detracts from the presentation here.

Now, let $\sigma_f^2 \defeq \E[\frac{f'(X)^2}{f(X)^2}]$ be the Fisher
information for the location in the family $\{P_\theta\}$, which is finite
when $f$ is log-concave and symmetric. We give particular attention to the
asymptotic relative efficiency (ARE) of estimators with respect to
asymptotically normal efficient estimators achieving the information
bound~\cite{VanDerVaart98}. In this case,
if $\{m(n), n \in \N\}$ is a sequence such that
\begin{equation*}
  \sqrt{m(n)} (\theta_n - \theta) \cd \normal(0, \sigma_f^2),
\end{equation*}
then the ARE of the estimator is~\cite[Def.~6.6.6]{LehmannCa98}
\begin{equation}
  \label{eqn:are-def}
  \ARE({\theta}_n) \defeq
  \liminf_{n\rightarrow \infty} \frac{m(n)}{n}.
\end{equation}
In the special case where there exists $V \in \mathbb R$ such that
\begin{equation*}
  m(n) R_n =
  m(n) \mathbb E_\theta\left({\theta}_n - \theta \right)^2 = V + o(1),
\end{equation*}
the ARE of ${\theta}_n$ is $\sigma_f^2/V$, so that $\theta_n$ requires a
sample $V / \sigma_f^2$-times larger than that of an efficient estimator for
comparable accuracy to the (information) efficient estimator.

\subsection*{Notation and basic assumptions}

To describe our results and make them formal, we require some additional
notation and one main assumption, which restricts the class of distributions
we consider.  We use the typical notation that
$F(x) = \int_{-\infty}^x f(t) dt$ is the cumulative distribution function
of the $X_i$, and we let
\begin{equation*}
  h(x) \triangleq \frac{f(x)}{1-F(x)} = \frac{f(x)}{F(-x)}
\end{equation*}
be the \emph{hazard} function (or the \emph{failure rate} or \emph{force of
  mortality}), which is monotone increasing as $f$ is
log-concave~\cite{bagnoli2005log}. Given the centrality of the median
to our efficiency bounds, it is unsurprising that the quantity
\begin{equation}
  \label{eq:eta_def}
  \eta(x) \triangleq \frac{f^2(x)}{F(x)(1-F(x))}
  \stackrel{(\star)}{=} \frac{f(x)f(-x)}{F(x)F(-x)}
\end{equation}
appears throughout our development. (Here equality~$(\star)$ is immediate by
the symmetry of $f$.)  For $p \in (0, 1)$ and $x = F^{-1}(p)$,
\begin{equation}
  \label{eqn:variance-quantiles}
  \frac{1}{\eta(x)} =
  \frac{1}{\eta(F^{-1}(p))}
  = \frac{p (1 - p)}{f(F^{-1}(p))^2}
\end{equation}
is of course the familiar asymptotic variance of the $p$th quantile of the sample $X_1,\ldots,X_n$ (cf.~\cite{VanDerVaart98}, Ch.~21).

For $f$ the normal density, classical results~\cite{Samford1953,
  hammersley1950estimating} show that $\eta(x)$ is a strictly
decreasing function of $|x|$, as we illustrate in Fig.~\ref{fig:eta}.
%
We consider log-concave symmetric distributions sharing this
property.  Specifically, we require the following.
\begin{assumption}
  \label{assump:failure_rate}
  The density $f$ is log-concave and symmetric.  Additionally, the origin $x
  = 0$ uniquely maximizes $\eta(x)$, and $\eta(x)$ is non-increasing in
  $|x|$.
\end{assumption}
Under this assumption,
\begin{equation*}
  4 f^2(x) \leq \eta(x) \leq \eta(0),
\end{equation*} 
%
where $\eta(0) = 4 f^2(0)$ is the asymptotic variance of the sample median
(Eq.~\eqref{eqn:variance-quantiles} at $p = 1/2$).  Combined with
log-concavity of $f(x)$, Assumption~\ref{assump:failure_rate} implies that
$\eta(x)$ vanishes as $|x|\rightarrow \infty$.  Several distributions
satisfy Assumption~\ref{assump:failure_rate}, including the generalized
normal distributions with a shape parameter between $1$ and $2$ (including
the normal and Laplace distributions). Symmetric log-concave distributions
failing Assumption~\ref{assump:failure_rate} include the uniform
distribution and the generalized normal distribution with shape parameter
greater than $2$. Some restriction on the class of distributions is
necessary to develop our results; indeed, in
Appendix~\ref{sec:uniform-weirdos} we provide a brief discussion on the
uniform distribution, where a one-step adaptive estimator with single bit
observations can achieve convergence rates faster than the familiar
$\sqrt{n}$ paramateric rate.

\begin{figure}
\begin{center}
\begin{tikzpicture}[scale = 0.6]
\begin{axis}[
width=8cm, height=6cm,
xmin = -3, xmax=3, 
restrict y to domain = 0:100,
ymin = 0,
ymax = 0.9,
samples=10, 
xlabel= $x$,
xtick={-2,-1,0,1,2},
xticklabels={-2,-1,0,1,2},
ytick={0,0.3989423,0.6366198},
yticklabels={0,$\frac{1}{\sqrt{2 \pi}}$,$2/\pi$},
line width=1.0pt,
mark size=1.5pt,
ymajorgrids,
xmajorgrids,
legend style= {at={(1,1)},anchor=north east,draw=black,fill=white,align=left}
]
\addplot[color = blue, solid, smooth] plot table [x = x, y = y, col sep=comma] {./Figs/eta.csv};
\addlegendentry{$\eta(x)$};
\addplot[domain = -5:5, samples = 50, color = red, solid, smooth]  {exp(-x^2/2) / sqrt(2*3.14159)};
\addlegendentry{$\phi(x)$};
\addplot[domain = -5:5, samples = 50, color = black, solid, dashed]  {4*exp(-x^2) / (2*3.14159)};
\addlegendentry{$4\phi^2(x)$};

\end{axis}
\end{tikzpicture}
%
\begin{tikzpicture}[scale = 0.6]
\begin{axis}[
width=8cm, height=6cm,
xmin = -4, xmax=4, 
restrict y to domain = -10:0,
ymin = -10,
samples=10, 
xlabel= $x$,
xtick={-3,-2,-1,0,1,2,3},
xticklabels={-3,-2,-1,0,1,2,3},
ytick={0,-0.45158,-0.919},
yticklabels={0,,},
line width=1.0pt,
mark size=1.5pt,
ymajorgrids,
xmajorgrids,
legend style= {at={(1,1)},anchor=north east,draw=black,fill=white,align=left}
]
\addplot[color = blue, solid, smooth] plot table [x = x, y = logy, col sep=comma] {./Figs/eta.csv};
\addlegendentry{$\log \eta(x)$};
\addplot[domain = -5:5, samples = 30, color = red, solid, smooth]  {-(x)^2/2 -0.9189};
\addlegendentry{$\log \phi(x)$};
\end{axis}
\end{tikzpicture}
\caption{
The function $\eta(x) = f^2(x) / F(x)F(-x)$ for $f(x) = \phi(x)$ the standard normal density.
\label{fig:eta}
}
\end{center}
\end{figure}
